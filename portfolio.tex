\documentclass[11pt]{scrartcl}
\usepackage[a4paper,margin=2.5cm]{geometry}
\setlength{\parindent}{0pt}
\setlength{\parskip}{4pt}
\usepackage[utf8]{inputenc}
\usepackage[brazil,portuguese]{babel}
% define styles for code listings
\usepackage[usenames,dvipsnames]{color}
\usepackage{listings}
\lstdefinestyle{cpp}{
    language=c++,%
    basicstyle=\tt\small,
    commentstyle=\color{Green},
    keywordstyle=\color{blue},
    stringstyle=\color{Orange},
    keepspaces=true,
    showstringspaces=false,
    frame=simple,%p
    breaklines=true,%p
    breakatwhitespace=true,%p
    numbers=left,%p
    tabsize=2,%p
    % numberstyle=\tiny,%p
    showspaces=false,
}
\lstdefinestyle{java}{
    language=java,%
    basicstyle=\tt\small,
    commentstyle=\color{Green},
    keywordstyle=\color{blue},
    stringstyle=\color{Orange},
    keepspaces=true,
    showstringspaces=false
}
% customize pdf document properties
\usepackage[pdftex]{hyperref}
\hypersetup{%
  bookmarksopen=false,
  colorlinks=true,
  citecolor=black,
  filecolor=black,
  linkcolor=black,
  urlcolor=blue
}

%------------------------------------------------------------------------------%
\begin{document}

\vspace*{\fill}

\hrule
\begin{center}
\Huge
\textsc{Portfólio de Introdução à Programação Competitiva}

\medskip
\Large
Lucas Caetano Possatti\\
\large
\texttt{lucas.c.possatti@gmail.com}

\bigskip

Professores:\\
\emph{Jefferson O. Andrade}\\
\emph{Flávio S. Lamas de Souza}

\bigskip

\today
\end{center}
\hrule

\vspace*{\fill}
\thispagestyle{empty}

\newpage

\setcounter{tocdepth}{2}
\tableofcontents

\newpage


% ~~~~~~~~~~~~~~~~~~~~~~~~~~~~~~~~~~~~~~~~~~~~~~~~~~~~~~~~~~~~~~~~~~~~~~~
\section{Introdução}

\subsection{Começando: Super Fáceis}

\subsubsection{UVa 10550 - Combination Lock}
Foi codificado o cálculo do total de ângulos, exatamente como descrito
no enunciado do problema.
\lstinputlisting[style=cpp]{src/combination.cpp}

\subsection{Começando: Fáceis}

\subsubsection{UVa 11799 - Horror Dash}
Para cada caso de teste, eu construo um array com as velocidades das ``criaturas assustadoras''. E desse array, eu procuro qual é a maior velocidade. Essa deve ser a velocidade mínima do palhaço.
\lstinputlisting[style=cpp]{src/uva11799-horror-dash/horror-dash.cpp}

\subsection{Começando: Médios}

\subsection{Problemas Ad Hoc}

TBD

% ~~~~~~~~~~~~~~~~~~~~~~~~~~~~~~~~~~~~~~~~~~~~~~~~~~~~~~~~~~~~~~~~~~~~~~~
\section{Bibliotecas e Estruturas de Dados}

\subsection{Estruturas de Dados Lineares}

\subsection{Estruturas de Dados Não Lineares}

\subsection{Bibliotecas Próprias: Grafos}

\subsection{Bibliotecas Próprias: Conjuntos Disjuntos}

\subsection{Bibliotecas Próprias: Árvores de Segmentos}

\subsection{Bibliotecas Próprias: Árvores de Fenwick}

TBD

% ~~~~~~~~~~~~~~~~~~~~~~~~~~~~~~~~~~~~~~~~~~~~~~~~~~~~~~~~~~~~~~~~~~~~~~~
\section{Paradigmas de Resolução de Problemas}

\subsection{Pesquisa Completa}

\subsection{Dividir e Conquistar}

\subsection{Algoritmos Gulosos}

\subsection{Programação Dinâmica}

TBD


% ~~~~~~~~~~~~~~~~~~~~~~~~~~~~~~~~~~~~~~~~~~~~~~~~~~~~~~~~~~~~~~~~~~~~~~~
\section{Grafos}

\subsection{Pesquisa em Grafos}

\subsection{Arvore Geradora Mínima}

\subsection{Caminho mais Curto de Origem Única}

\subsection{Caminho mais Curto de Todos os Pares}

\subsection{Fluxo em Rede}

\subsection{Grafos Especiais}

TBD


% ~~~~~~~~~~~~~~~~~~~~~~~~~~~~~~~~~~~~~~~~~~~~~~~~~~~~~~~~~~~~~~~~~~~~~~~
\section{Matemática}

\subsection{Combinatória}

\subsection{Teoria dos Números}

\subsection{Probabilidade}

\subsection{Identificação de Ciclos}

\subsection{Teoria dos Jogos}

TBD


% ~~~~~~~~~~~~~~~~~~~~~~~~~~~~~~~~~~~~~~~~~~~~~~~~~~~~~~~~~~~~~~~~~~~~~~~
\section{Processamento de Strings}

\subsection{Correspondência de String}

\subsection{Processamento de Strings com Programação Dinâmica}

\subsection{Suffix Trie/Tree/Array}

TBD


% ~~~~~~~~~~~~~~~~~~~~~~~~~~~~~~~~~~~~~~~~~~~~~~~~~~~~~~~~~~~~~~~~~~~~~~
\section{Geometria Computacional}

\subsection{Objetos Geométricos Básicos com Bibliotecas}

\subsection{Algoritmos de Polígonos com Bibliotecas}

TBD


% ~~~~~~~~~~~~~~~~~~~~~~~~~~~~~~~~~~~~~~~~~~~~~~~~~~~~~~~~~~~~~~~~~~~~~~
\section{Tópicos mais Avançados}

\subsection{Técnicas mais Avançadas de Pesquisa}

\subsection{Técnicas mais Avançadas de Programação Dinâmica}

\subsection{Decomposição de Problemas}

TBD


\end{document}

%%% Local Variables:
%%% mode: latex
%%% ispell-local-dictionary: "brasileiro"
%%% TeX-master: t
%%% End:
